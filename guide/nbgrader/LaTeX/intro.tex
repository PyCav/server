nbgrader\footnote{\url{https://github.com/jupyter/nbgrader}} is a system which allows for the use of Jupyter\footnote{\url{http://jupyter.org/}} notebooks as a form of assessment material. 

Lecturers (or at least, their lackeys\footnote{\url{https://pycav.github.io/about/}}) can create a set of exercises contained within these notebooks. They can be distributed to students using the JupyterHub\footnote{\url{https://github.com/jupyterhub/jupyterhub}} server. 

From there, students can complete the computational exercises and return them (hopefully completed) to the lecturers. The lecturers can autograde the exercises and manually mark aspects which do not lend themselves easily to a machine marking them (such as graphs).

This document outlines how the PyCav project set up nbgrader for use within the Physics courses within the Natural Sciences Tripos.

Another source of documentation for nbgrader can be found here: \url{http://nbgrader.readthedocs.io/}.